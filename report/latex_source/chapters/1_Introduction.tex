Sales forecasting is crucial for retailers in order to predict the future demand of products, which can in turn increase profit by ensuring the replenishment of the necessary supply to meet future demand, as well as minimizing product waste \cite{c13}.
However, the volatility in demand makes sales forecasting a challenging problem \cite{c14}.
This volatility is dependent on many external factors, such as holidays, events, price, and promotions.
Hence, it is necessary to take into account the effects of these external factors when forecasting future sales.
In some approaches, the manager guides the forecasting model based on his/her knowledge of the external factors by using fuzzy logic.
However, recent studies have focused on creating models that are able to directly take into account the effects of these external factors.

In this paper, we will compare the performance of three models, which consist of Long Short-Term Memory (LSTM), Multi-Layer Perceptron (MLP), and LightGBM (LGBM) models.
Furthermore, the LSTM model and the best performing model between the MLP and LGBM models will be used to construct a hybrid model to check if it is able to improve the performance compared to the two singular models.
